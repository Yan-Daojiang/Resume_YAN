%%%%%%%%%%%%%%%%%
% This is an example CV created using altacv.cls (v1.1.5, 1 December 2018) written by
% Neeraj Giri (giri492neeraj@gmail.com), based on the
% Cv created by BusinessInsider at http://www.businessinsider.my/a-sample-resume-for-marissa-mayer-2016-7/?r=US&IR=T
%
%% It may be distributed and/or modified under the
%% conditions of the LaTeX Project Public License, either version 1.3
%% of this license or (at your option) any later version.
%% The latest version of this license is in
%%    http://www.latex-project.org/lppl.txt
%% and version 1.3 or later is part of all distributions of LaTeX
%% version 2003/12/01 or later.
%%%%%%%%%%%%%%%%

%% If you are using \orcid or academicons
%% icons, make sure you have the academicons
%% option here, and compile with XeLaTeX
%% or LuaLaTeX.
% \documentclass[10pt,a4paper,academicons]{altacv}

%% Use the "normalphoto" option if you want a normal photo instead of cropped to a circle
% \documentclass[10pt,a4paper,normalphoto]{altacv}

\documentclass[10pt,a4paper,ragged2e]{altacv}

%% AltaCV uses the fontawesome and academicon fonts
%% and packages.
%% See texdoc.net/pkg/fontawecome and http://texdoc.net/pkg/academicons for full list of symbols. You MUST compile with XeLaTeX or LuaLaTeX if you want to use academicons.

% Change the page layout if you need to
\geometry{left=1cm,right=9cm,marginparwidth=6.8cm,marginparsep=1.2cm,top=1.25cm,bottom=1.25cm}
\usepackage{ctex}   % 如果想让模板支持中文,引入ctex宏包
\usepackage{hyperref}
\hypersetup{hidelinks} % 隐藏超链接的小框

% Change the font if you want to, depending on whether
% you're using pdflatex or xelatex/lualatex
\ifxetexorluatex
  % If using xelatex or lualatex:
  \setmainfont{Carlito}
\else
  % If using pdflatex:
  \usepackage[utf8]{inputenc}
  \usepackage[T1]{fontenc}
  \usepackage[default]{lato}
\fi

% Change the colours if you want to
\definecolor{VividPurple}{HTML}{3E0097}
\definecolor{SlateGrey}{HTML}{2E2E2E}
\definecolor{LightGrey}{HTML}{666666}
\colorlet{heading}{VividPurple}
\colorlet{accent}{VividPurple}
\colorlet{emphasis}{SlateGrey}
\colorlet{body}{LightGrey}

% Change the bullets for itemize and rating marker
% for \cvskill if you want to
\renewcommand{\itemmarker}{{\small\textbullet}}
\renewcommand{\ratingmarker}{\faCircle}

%% sample.bib contains your publications
%% \addbibresource{sample.bib}

\begin{document}
\name{颜道江}    % 
\tagline{个人简历}

%\tagline{Software and Data enthusiast, Technical Hoarder \& Proud Geek}


\photo{4.0cm}{pic/yan}
\personalinfo{%
  % Not all of these are required!
  % You can add your own with \printinfo{symbol}{detail}
  \href{mailto:yan\_daojiang@163.como}{\email{yan\_daojiang@163.com}}
  \phone{+86-18186451525}
  \wechat{ydj1373836603}
  \QQ{1373836603}
  
  % \linkedin{linkedin.com/in/}
  %\href{http://yandaojiang.com}{\homepage{yandaojiang.com}}
  \href{https://github.com/yan-daojiang}{\github{github.com/yan-daojiang}}
  % \href{https://blog.csdn.net/qq_40892332}{\homepage{CSDN}}
  \location{湖北, 武汉}
  \mailaddress{湖北省武汉市洪山区,文秀街升升公寓C栋,430070}

}

%% Make the header extend all the way to the right, if you want.
\begin{fullwidth}
\makecvheader
\end{fullwidth}

%% Depending on your tastes, you may want to make fonts of itemize environments slightly smaller
\AtBeginEnvironment{itemize}{\small}

%% Provide the file name containing the sidebar contents as an optional parameter to \cvsection.
%% You can always just use \marginpar{...} if you do
%% not need to align the top of the contents to any
%% \cvsection title in the "main" bar.
\cvsection[page1sidebar]{PROJECTS}

\cvevent{FAIR SHARE}{学期实践 }{Dec 2019 -- Jan 2020}{湖北, 武汉 }
\begin{itemize}
\item\textbf{项目简介:\,}FAIR    \, SHARE是一个局域网文件共享系统.该项目基于TCP/IP通信协议,使用C++ QT框架开发,主要功能为实现局域网内的P2P文件共享,支持断点续传等功能.
\item \textbf{主要技术:\,}\emph{QT框架\quad TCP/IP通信协议\quad MySQL数据库}
\item 个人主要负责需求分析,数据库设计实现以及服务端开发.

\end{itemize}

\divider

\cvevent{图书馆管理系统}{课程实践}{NOV 2019 -- DEC 2019}{}
\begin{itemize}
\item\textbf{项目简介:\,} 该项目为数据库系统基本原理课程实践,要求实现一个图书馆管理系统.
\item \textbf{主要技术:\,}\emph{PHP\quad Bootstrap框架}
\item 个人主要负责数据库逻辑设计与前端开发.
\end{itemize}

\divider

\cvevent{极坐标系下叶片特征提取及基于 K 近邻 的叶片类型判别}{数学建模练习}{July 2019}{}
\begin{itemize}
\item\textbf{项目简介:\,} 该项目主要通过在极坐标系下对叶片的几何特征进行提取,然后利用K近邻算法实现对叶片的分类.项目全部通过python实现,测试集准确率可达90\%以上.
\item \textbf{主要技术:\,}\emph{K近邻算法}
\item 个人负责所有特征提取,算法实现与优化.
\end{itemize}

\divider

\cvevent{欢乐连连看}{课程实践}{June 2019}{}
\begin{itemize}
	\item\textbf{项目简介:\,} 该项目为数据结构与算法课程实践,主要基于C++ MFC开发一个连连看小游戏.
	\item \textbf{主要技术:\,}\emph{数据结构-线性结构 \quad MFC}
	\item 个人项目,负责所有设计与实现.
\end{itemize}

\clearpage

\cvsection[page2sidebar]{Projects}
\cvevent{基于时间序列的电商销量预测 }{数学建模竞赛}{May 2019 }{湖北, 武汉}
\begin{itemize}
\item\textbf{项目简介:\,}该项目主要通过对电商以往销量进行可视化并构建ARIMA模型进行短期销量预测.
\item \textbf{主要技术:\,}\emph{ARIMA模型\quad 数据可视化 \quad 统计分析}
\item 个人负责数据可视化工作以及通过Python编程构建ARIMA模型进行短期销量预测.
\end{itemize}

\divider

\cvevent{景区管理系统}{课程实践}{April 2019}{}
\begin{itemize}
\item\textbf{项目简介:\,}该项目为数据结构与算法课程实践项目,主要通过C++编程实现各类图算法.
\item \textbf{主要技术:\,}\emph{C++程序设计 \quad 图算法}
\item 个人项目,负责所有功能点的分析设计与实现工作.
\end{itemize}

\divider




% \begin{itemize}
% \item Joined the co

%\begin{flushleft}
%	\emph{Jan 19th, 2020}
%\end{flushleft}
\begin{flushright}
	\tiny{更新于\today \qquad 颜道江}
\end{flushright}

\clearpage

\end{document}


